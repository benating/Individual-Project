\chapter{Background}
Waves exhibit many characteristics when travelling through different mediums. 

\section{Dispersion relation}
A dispersion relation relates the wavenumber of a wave to its frequency. We
will see that this relation is of utmost importance when discussing lattices as
only waves of certain frequencies are transmitted by the lattice.

\subsection{1D lattice}
Let us start our discussion with the most basic one-dimensional case. In this
case, we have masses of equal mass, $m$ spread out evenly across one dimension.
All neighbouring masses are connected by an elastic force which scales
proportionally with distance, i.e. $F = kx$ for some $k$.

% TODO: Add discussion about transverse waves into appendix?
\textit{Note}: It is very useful to think of this system as a mass-spring
system. For simplicity, we will be considering longitudinal waves, however the
same analysis can be done for transverse waves as we show in the appendix.

With this one-dimensional case set up, we now want to find out what forms of
waves it allows to propagate. We do this by examining the elementary unit of
the lattice which can be repeated by translation to form the full lattice.

So let us consider three masses side-by-side in our lattice, $M_{n-1}$, $M_n$
and $M_{n+1}$ which are $y_{n-1}$, $y_n$ and $y_{n+1}$ displacement away from
their equilibrium positions. By focusing on the centre mass and considering
only nearest neighbour interactions, by Hooke's Law, we have that the force on
$M_n$

\begin{align}
  F_{n}=\sum F=k\left(y_{n-1}+y_{n+1}-2y_{n}\right) \label{eq:HL}
\end{align}

At the same time, we have by Newton's 2nd Law that

\begin{align}
  F_{n}=m\frac{d^{2}}{dt^{2}}y_{n}
\end{align}

Taking $y_n$ to be the general wave solution

\begin{align}
  &y_{n}=\hat{y}_{n}e^{-i\Omega t} \\
  \Rightarrow &\frac{d^{2}}{dt^{2}}y_{n}=-\Omega^{2}y_{n} \\
  \Rightarrow &F_{n}=-m\Omega^{2}y_{n}a \label{eq:N2L}
\end{align}

where $\Omega$ is the frequency of the wave and $\hat{y}_{n}$ is the complex
amplitude.

Therefore, we have by \eqref{eq:HL} and \eqref{eq:N2L} that

\begin{align}
  &-m\Omega^{2}y_{n}=k\left(y_{n-1}+y_{n+1}-2y_{n}\right) \\
  \Rightarrow &\left(-\frac{m}{k}\Omega^{2}+2\right)y_{n}-y_{n-1}-y_{n+1}=0
    \label{eq:HLN2L}
\end{align}

Now since we have that $M_{n-1}$ and $M_{n+1}$ are equidistant from $M_{n}$ at
equilibrium, we can describe the phase shift in the wave solution as

\begin{align}
  y_{n-1}=e^{-i\kappa}y_n,\ \ y_{n+1}=e^{i\kappa}y_n
\end{align}

Hence, from \eqref{eq:HLN2L}, we see that

\begin{align}
  &\left(-\frac{m}{k}\Omega^{2}+2-e^{-i\kappa}-e^{i\kappa}\right)y_{n}=0 \\
  \Rightarrow &\cos\left(\kappa\right)=\frac{2m}{k}\Omega^{2}+4
\end{align}

\subsection{2D square lattice}

\subsection{2D square lattice with two masses}

\subsection{2D hexagonal lattice}

\subsection{2D hexagonal lattice with two masses}
