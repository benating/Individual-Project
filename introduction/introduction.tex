\chapter{Introduction}
Since the earliest days of human history, we have sought to better understand
the natural world around us and so create technologies to improve our lives. At
first by utilising what nature already has to offer, but later on developing
materials and devices which have desirable properties not found in nature.

Metamaterials are precisely that; materials engineered to possess some property
not found in naturally occurring materials \cite{briefintro}. Usually these
materials are composed of smaller units which are arranged in a repeating
pattern. They derive their properties not from the intrinsic properties of the
base materials, but from the periodic structure introduced.

One of the first major breakthroughs which the study of metamaterials led to
was the development of materials with a negative refractive index. Though the
possibility of creating such a material was theoretically predicted in 1967
\cite{negrefrac}, it was only experimentally verified in 2001
\cite{negrefracex}.

\section{Practical applications}
It is important to discuss the real-life applications which this project can
lead to as the study of metamaterials has the potential to solve many modern
engineering problems.

\begin{itemize}
\item Developing a cloak of invisibility \cite{emcloak}.
      \begin{itemize}
      \item Obvious use cases in military.
      \item Protect things from electromagnetic radiation.
      \end{itemize}
\item Constructing antennae to focus energy into narrow beams.
      \cite{diremi,antennasol}
      \begin{itemize}
      \item Gather wave energy from deep in the oceans and concentrate them to
            be harvested.
      \item Channel sound energy, e.g. from train tracks, to be dissipated
            elsewhere which decreases noise pollution.
      \end{itemize}
\item Producing devices which operate on the terahertz ($1THz=10^{12}Hz$) range.
      \cite{THz}
      \begin{itemize}
      \item Can be used in medical imaging due to terahertz radiation being
            able to penetrate thin materials but not thicker objects. Also, it
            is non-ionising radiation and so does not damage living cells
            unlike X-rays.
      \item For surveillance, especially security screening as many materials
            of interest have unique spectral fingerprints in the THz range
            \cite{Thzsec}.
      \end{itemize}
\item Developing near perfect electromagnetic absorbers \cite{absorbing}.
\item Developing perfect lenses \cite{negrefraclens}.
\end{itemize}

\section{Objectives}
In this project, our main goal is to be able to geometrically navigate waves
around sharp and gentle bends by constructing a semi-infinite lattice composed
of two structurally different materials where the waves will travel along the
boundary between the two materials. In our case, we will consider elastic waves
travelling through a mass-spring system.

In particular, we will be carrying out the following:

\begin{enumerate}
\item Build an intuition to study dispersion relations by first finding them
      for simpler models.
\item Perturb structures of elementary cells of infinite lattices to open up
      band gaps.
\item Construct a semi-infinite lattice formed of ribbons created from two
      differently perturbed elementary cells and observe wave behaviour at the
      boundary.
\item Simulate scattering to confirm direction of dissipation of energy.
\end{enumerate}


