\chapter{Introduction}
Many advances in the realm of photonics, phononics and even next-generation
electronics come from the ability to manipulate and control wave
energy.\cite{gravwaves,emcloak,diremi,antennasol,THz,absorbing,toposplit,negrefraclens}
Most of technologies depend on being able to direct wave propagation as well as
split and merge waves on demand without loss of energy. Many recent advances
have come from the application of topological concepts to guided wave
propagation.\cite{singlevalley,qshe,valleyhall,2dtopophoto} One such phenomenon
is the quantum valley-Hall effect which exploits valley states to achieve
topologically protected edge modes in a lattice of electrons by creating
time-reversal symmetric valley-Hall
insulators,\cite{allsitopo,dielectopo,sonicvalley} which can be used to direct
wave propagation. Hence in this work, we will be investigating 2-dimensional
mass-spring systems that emulate the quantum valley-Hall effect. This provides
the ability to experiment, explore and simulate other novel behaviours and
properties in a simpler system. Though we do not break time reversal symmetry
in our model (i.e. we have two way propagation), we break the physical symmetry
of our lattices to get as close as we can to their effects.

\section{Objectives}
In this project, our main goal is to emulate the quantum valley Hall
effect\cite{mos2, valleyhall} in a model mass-spring system. We will then use
these systems to geometrically navigate waves around different bends by
constructing a finite lattice composed of two topologically different materials
where the elastic waves will travel along the boundary between the two
materials.

In particular, we will be carrying out the following:

\begin{enumerate}
\item Build an intuition to study dispersion relations by first finding them
      for simpler models. (Chapter \ref{disperrel})
\item Perturb structures of elementary cells of infinite lattices to open up
      band gaps. Then construct a semi-infinite lattice formed of ribbons
      created from two differently perturbed elementary cells and observe wave
      behaviour at the boundary. (Chapter \ref{perturbed})
\item Simulate scattering to confirm direction of dissipation of energy.
      (Chapter \ref{scattering})
\end{enumerate}

Though the construction and fabrication of metamaterials involve much more
considerations in terms of material science, in our case, we are experimenting
on toy model systems computationally to discover new properties and behaviours.
This will allow us to investigate new phenomena in the context of a much
simpler system. And if we are able to experimentally verify the observations we
see in our simulations as well as corroborate results found in other more
complex models, we can then go on to perform more interesting further work on
this toy model.

\section{Motivation}
There are many reasons to want to create metamaterials which can control the
directions of electromagnetic or acoustic waves. As we discuss in Chapter
\ref{applications}, metamaterials have the capability to solve many engineering
problems. However, metamaterials manufacturing presents many challenges
including requiring machines which are capable of reliably working on the
micro- and nano-scale.\cite{metamanu} Therefore, being able to model and
investigate their properties computationally will allow us to decrease
prototyping time as well as discover new features which might be of interest
efficiently.
