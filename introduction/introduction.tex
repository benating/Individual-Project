\chapter{Introduction}

\section{Objectives}
In this project, our main goal is to be able to geometrically navigate waves
around sharp and gentle bends by constructing a semi-infinite lattice composed
of two structurally different materials where the waves will travel along the
boundary between the two materials. In our case, we will consider elastic waves
travelling through a mass-spring system.

In particular, we will be carrying out the following:

\begin{enumerate}
\item Build an intuition to study dispersion relations by first finding them
      for simpler models. (Chapter \ref{disperrel})
\item Perturb structures of elementary cells of infinite lattices to open up
      band gaps. Then construct a semi-infinite lattice formed of ribbons
      created from two differently perturbed elementary cells and observe wave
      behaviour at the boundary. (Chapter \ref{perturbed})
\item Simulate scattering to confirm direction of dissipation of energy.
      (Chapter \ref{scattering})
\end{enumerate}

Though the construction and fabrication of metamaterials involve much more
considerations in terms of material science, in our case, we are experimenting
on toy model systems computationally to discover new properties and behaviours.
This will allow us to investigate new phenomena in the context of a much
simpler system. And if we are able to experimentally verify the observations we
see in our simulations, we can then go on to perform more interesting further
work on this toy model.

\section{Motivation}
There are many reasons to want to create metamaterials which can control the
directions of electromagnetic or acoustic waves. As we discuss in Chapter
\ref{applications}, metamaterials have the capability to solve many engineering
problems. However, metamaterials manufacturing presents many challenges
including requiring machines which are capable of reliably working on the
micro- and nano-scale.\cite{metamanu} Therefore, being able to model and
investigate their properties computationally will allow us to decrease
prototyping time as well as discover new features which might be of interest
efficiently.
